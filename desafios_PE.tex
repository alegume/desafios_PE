% Options for packages loaded elsewhere
\PassOptionsToPackage{unicode}{hyperref}
\PassOptionsToPackage{hyphens}{url}
\PassOptionsToPackage{dvipsnames,svgnames,x11names}{xcolor}
%
\documentclass[
  letterpaper,
  DIV=11,
  numbers=noendperiod]{scrartcl}

\usepackage{amsmath,amssymb}
\usepackage{iftex}
\ifPDFTeX
  \usepackage[T1]{fontenc}
  \usepackage[utf8]{inputenc}
  \usepackage{textcomp} % provide euro and other symbols
\else % if luatex or xetex
  \usepackage{unicode-math}
  \defaultfontfeatures{Scale=MatchLowercase}
  \defaultfontfeatures[\rmfamily]{Ligatures=TeX,Scale=1}
\fi
\usepackage{lmodern}
\ifPDFTeX\else  
    % xetex/luatex font selection
\fi
% Use upquote if available, for straight quotes in verbatim environments
\IfFileExists{upquote.sty}{\usepackage{upquote}}{}
\IfFileExists{microtype.sty}{% use microtype if available
  \usepackage[]{microtype}
  \UseMicrotypeSet[protrusion]{basicmath} % disable protrusion for tt fonts
}{}
\makeatletter
\@ifundefined{KOMAClassName}{% if non-KOMA class
  \IfFileExists{parskip.sty}{%
    \usepackage{parskip}
  }{% else
    \setlength{\parindent}{0pt}
    \setlength{\parskip}{6pt plus 2pt minus 1pt}}
}{% if KOMA class
  \KOMAoptions{parskip=half}}
\makeatother
\usepackage{xcolor}
\setlength{\emergencystretch}{3em} % prevent overfull lines
\setcounter{secnumdepth}{-\maxdimen} % remove section numbering
% Make \paragraph and \subparagraph free-standing
\ifx\paragraph\undefined\else
  \let\oldparagraph\paragraph
  \renewcommand{\paragraph}[1]{\oldparagraph{#1}\mbox{}}
\fi
\ifx\subparagraph\undefined\else
  \let\oldsubparagraph\subparagraph
  \renewcommand{\subparagraph}[1]{\oldsubparagraph{#1}\mbox{}}
\fi

\usepackage{color}
\usepackage{fancyvrb}
\newcommand{\VerbBar}{|}
\newcommand{\VERB}{\Verb[commandchars=\\\{\}]}
\DefineVerbatimEnvironment{Highlighting}{Verbatim}{commandchars=\\\{\}}
% Add ',fontsize=\small' for more characters per line
\usepackage{framed}
\definecolor{shadecolor}{RGB}{241,243,245}
\newenvironment{Shaded}{\begin{snugshade}}{\end{snugshade}}
\newcommand{\AlertTok}[1]{\textcolor[rgb]{0.68,0.00,0.00}{#1}}
\newcommand{\AnnotationTok}[1]{\textcolor[rgb]{0.37,0.37,0.37}{#1}}
\newcommand{\AttributeTok}[1]{\textcolor[rgb]{0.40,0.45,0.13}{#1}}
\newcommand{\BaseNTok}[1]{\textcolor[rgb]{0.68,0.00,0.00}{#1}}
\newcommand{\BuiltInTok}[1]{\textcolor[rgb]{0.00,0.23,0.31}{#1}}
\newcommand{\CharTok}[1]{\textcolor[rgb]{0.13,0.47,0.30}{#1}}
\newcommand{\CommentTok}[1]{\textcolor[rgb]{0.37,0.37,0.37}{#1}}
\newcommand{\CommentVarTok}[1]{\textcolor[rgb]{0.37,0.37,0.37}{\textit{#1}}}
\newcommand{\ConstantTok}[1]{\textcolor[rgb]{0.56,0.35,0.01}{#1}}
\newcommand{\ControlFlowTok}[1]{\textcolor[rgb]{0.00,0.23,0.31}{#1}}
\newcommand{\DataTypeTok}[1]{\textcolor[rgb]{0.68,0.00,0.00}{#1}}
\newcommand{\DecValTok}[1]{\textcolor[rgb]{0.68,0.00,0.00}{#1}}
\newcommand{\DocumentationTok}[1]{\textcolor[rgb]{0.37,0.37,0.37}{\textit{#1}}}
\newcommand{\ErrorTok}[1]{\textcolor[rgb]{0.68,0.00,0.00}{#1}}
\newcommand{\ExtensionTok}[1]{\textcolor[rgb]{0.00,0.23,0.31}{#1}}
\newcommand{\FloatTok}[1]{\textcolor[rgb]{0.68,0.00,0.00}{#1}}
\newcommand{\FunctionTok}[1]{\textcolor[rgb]{0.28,0.35,0.67}{#1}}
\newcommand{\ImportTok}[1]{\textcolor[rgb]{0.00,0.46,0.62}{#1}}
\newcommand{\InformationTok}[1]{\textcolor[rgb]{0.37,0.37,0.37}{#1}}
\newcommand{\KeywordTok}[1]{\textcolor[rgb]{0.00,0.23,0.31}{#1}}
\newcommand{\NormalTok}[1]{\textcolor[rgb]{0.00,0.23,0.31}{#1}}
\newcommand{\OperatorTok}[1]{\textcolor[rgb]{0.37,0.37,0.37}{#1}}
\newcommand{\OtherTok}[1]{\textcolor[rgb]{0.00,0.23,0.31}{#1}}
\newcommand{\PreprocessorTok}[1]{\textcolor[rgb]{0.68,0.00,0.00}{#1}}
\newcommand{\RegionMarkerTok}[1]{\textcolor[rgb]{0.00,0.23,0.31}{#1}}
\newcommand{\SpecialCharTok}[1]{\textcolor[rgb]{0.37,0.37,0.37}{#1}}
\newcommand{\SpecialStringTok}[1]{\textcolor[rgb]{0.13,0.47,0.30}{#1}}
\newcommand{\StringTok}[1]{\textcolor[rgb]{0.13,0.47,0.30}{#1}}
\newcommand{\VariableTok}[1]{\textcolor[rgb]{0.07,0.07,0.07}{#1}}
\newcommand{\VerbatimStringTok}[1]{\textcolor[rgb]{0.13,0.47,0.30}{#1}}
\newcommand{\WarningTok}[1]{\textcolor[rgb]{0.37,0.37,0.37}{\textit{#1}}}

\providecommand{\tightlist}{%
  \setlength{\itemsep}{0pt}\setlength{\parskip}{0pt}}\usepackage{longtable,booktabs,array}
\usepackage{calc} % for calculating minipage widths
% Correct order of tables after \paragraph or \subparagraph
\usepackage{etoolbox}
\makeatletter
\patchcmd\longtable{\par}{\if@noskipsec\mbox{}\fi\par}{}{}
\makeatother
% Allow footnotes in longtable head/foot
\IfFileExists{footnotehyper.sty}{\usepackage{footnotehyper}}{\usepackage{footnote}}
\makesavenoteenv{longtable}
\usepackage{graphicx}
\makeatletter
\def\maxwidth{\ifdim\Gin@nat@width>\linewidth\linewidth\else\Gin@nat@width\fi}
\def\maxheight{\ifdim\Gin@nat@height>\textheight\textheight\else\Gin@nat@height\fi}
\makeatother
% Scale images if necessary, so that they will not overflow the page
% margins by default, and it is still possible to overwrite the defaults
% using explicit options in \includegraphics[width, height, ...]{}
\setkeys{Gin}{width=\maxwidth,height=\maxheight,keepaspectratio}
% Set default figure placement to htbp
\makeatletter
\def\fps@figure{htbp}
\makeatother

<script src="desafios_PE_files/libs/htmlwidgets-1.6.2/htmlwidgets.js"></script>
<script src="desafios_PE_files/libs/viz-0.3/viz.js"></script>
<link href="desafios_PE_files/libs/DiagrammeR-styles-0.2/styles.css" rel="stylesheet" />
<script src="desafios_PE_files/libs/grViz-binding-0.9.0/grViz.js"></script>
\KOMAoption{captions}{tableheading}
\makeatletter
\makeatother
\makeatletter
\makeatother
\makeatletter
\@ifpackageloaded{caption}{}{\usepackage{caption}}
\AtBeginDocument{%
\ifdefined\contentsname
  \renewcommand*\contentsname{Índice}
\else
  \newcommand\contentsname{Índice}
\fi
\ifdefined\listfigurename
  \renewcommand*\listfigurename{Lista de Figuras}
\else
  \newcommand\listfigurename{Lista de Figuras}
\fi
\ifdefined\listtablename
  \renewcommand*\listtablename{Lista de Tabelas}
\else
  \newcommand\listtablename{Lista de Tabelas}
\fi
\ifdefined\figurename
  \renewcommand*\figurename{Figura}
\else
  \newcommand\figurename{Figura}
\fi
\ifdefined\tablename
  \renewcommand*\tablename{Tabela}
\else
  \newcommand\tablename{Tabela}
\fi
}
\@ifpackageloaded{float}{}{\usepackage{float}}
\floatstyle{ruled}
\@ifundefined{c@chapter}{\newfloat{codelisting}{h}{lop}}{\newfloat{codelisting}{h}{lop}[chapter]}
\floatname{codelisting}{Listagem}
\newcommand*\listoflistings{\listof{codelisting}{Lista de Listagens}}
\makeatother
\makeatletter
\@ifpackageloaded{caption}{}{\usepackage{caption}}
\@ifpackageloaded{subcaption}{}{\usepackage{subcaption}}
\makeatother
\makeatletter
\@ifpackageloaded{tcolorbox}{}{\usepackage[skins,breakable]{tcolorbox}}
\makeatother
\makeatletter
\@ifundefined{shadecolor}{\definecolor{shadecolor}{rgb}{.97, .97, .97}}
\makeatother
\makeatletter
\makeatother
\makeatletter
\makeatother
\ifLuaTeX
\usepackage[bidi=basic]{babel}
\else
\usepackage[bidi=default]{babel}
\fi
\babelprovide[main,import]{brazilian}
% get rid of language-specific shorthands (see #6817):
\let\LanguageShortHands\languageshorthands
\def\languageshorthands#1{}
\ifLuaTeX
  \usepackage{selnolig}  % disable illegal ligatures
\fi
\IfFileExists{bookmark.sty}{\usepackage{bookmark}}{\usepackage{hyperref}}
\IfFileExists{xurl.sty}{\usepackage{xurl}}{} % add URL line breaks if available
\urlstyle{same} % disable monospaced font for URLs
\hypersetup{
  pdflang={pt-BR},
  colorlinks=true,
  linkcolor={blue},
  filecolor={Maroon},
  citecolor={Blue},
  urlcolor={Blue},
  pdfcreator={LaTeX via pandoc}}

\author{}
\date{}

\begin{document}
\ifdefined\Shaded\renewenvironment{Shaded}{\begin{tcolorbox}[interior hidden, boxrule=0pt, frame hidden, breakable, sharp corners, borderline west={3pt}{0pt}{shadecolor}, enhanced]}{\end{tcolorbox}}\fi

Probabilidade e Estatística - Desafio 1

\begin{figure}

{\centering \includegraphics[width=2.08333in,height=\textheight]{images/ita.jpg}

}

\end{figure}

\textbf{Instituto Tecnológico de Aeronáutica}

\textbf{Professor}: Mauri Aparecido de Oliveira

\textbf{Aluno}: Alexandre A. A. M. de Abreu

\hypertarget{paciente}{%
\subsection{Paciente}\label{paciente}}

Um paciente apresenta um conjunto de sintomas que podem ser enquadrados
em dois tipos de doenças, conforme apresentado a seguir.

\textbf{DOENÇA A:} Grave. Deve ser medicada senão as consequências são
graves. Os remédios provocam efeitos colaterais.\\
\textbf{DOENÇA B:} Sem nenhuma gravidade. A cura ocorre naturalmente.
Exige apenas repouso.

Se a doença for \textbf{A} e medicar, a cura ocorrerá e a este evento
associamos um valor de \textbf{+5.000}. Entretanto, se não medicar o
quadro se complicará e a este evento associamos o valor
\textbf{--10.000}. A primeira análise da situação conduz a uma
probabilidade de \textbf{10\%} da doença ser \textbf{A}. Se a doença for
\textbf{B} e medicar, aparecerão efeitos colaterais indesejáveis e a
este evento associamos o valor \textbf{--300}. A não medicar e esperar
que os sintomas passem sozinhos associamos o valor \textbf{+500}.

\begin{center}\rule{0.5\linewidth}{0.5pt}\end{center}

\hypertarget{a-utilizando-o-crituxe9rio-do-ve-recomende-o-procedimento-a-seguir.}{%
\subsubsection{A) Utilizando o critério do VE, recomende o procedimento
a
seguir.}\label{a-utilizando-o-crituxe9rio-do-ve-recomende-o-procedimento-a-seguir.}}

Inicialmente, as bibliotecas necessárias são importadas e definem-se os
dados para o cálculo do Valor Esperado (VE) e para a criação da árvore
de decisão.

\begin{Shaded}
\begin{Highlighting}[]
\CommentTok{\# Bibliotecas necessárias para criar árvores }
\FunctionTok{library}\NormalTok{(data.tree)}
\FunctionTok{library}\NormalTok{(yaml)}
\FunctionTok{library}\NormalTok{(dplyr)}
\FunctionTok{library}\NormalTok{(usethis)}
\FunctionTok{library}\NormalTok{(devtools)}
\FunctionTok{library}\NormalTok{(DiagrammeR)}

\NormalTok{pa }\OtherTok{\textless{}{-}} \FloatTok{0.1}       \CommentTok{\# Probabilidade da doença ser A}
\NormalTok{pb }\OtherTok{\textless{}{-}} \FloatTok{0.9}       \CommentTok{\# Probabilidade da doença ser B}
\NormalTok{v\_am }\OtherTok{\textless{}{-}} \DecValTok{5000}    \CommentTok{\# Valor do evento: doença A e medicar}
\NormalTok{v\_anm }\OtherTok{\textless{}{-}} \SpecialCharTok{{-}}\DecValTok{10000} \CommentTok{\# Valor do evento: doença A e NAO medicar}
\NormalTok{v\_bm }\OtherTok{\textless{}{-}} \SpecialCharTok{{-}}\DecValTok{300}    \CommentTok{\# Valor do evento: doença B e medicar}
\NormalTok{v\_bnm }\OtherTok{\textless{}{-}} \DecValTok{500}    \CommentTok{\# Valor do evento: doença B e NAO medicar}
\NormalTok{VE.SIM }\OtherTok{\textless{}{-}}\NormalTok{ pa }\SpecialCharTok{*}\NormalTok{ v\_am }\SpecialCharTok{+}\NormalTok{ pb }\SpecialCharTok{*}\NormalTok{ v\_bm}
\NormalTok{VE.NAO }\OtherTok{\textless{}{-}}\NormalTok{ pa }\SpecialCharTok{*}\NormalTok{ v\_anm }\SpecialCharTok{+}\NormalTok{ pb }\SpecialCharTok{*}\NormalTok{ v\_bnm}
\NormalTok{VE }\OtherTok{=} \FunctionTok{max}\NormalTok{(VE.SIM, VE.NAO) }\CommentTok{\# Define o VE}
\NormalTok{VE}
\end{Highlighting}
\end{Shaded}

\begin{verbatim}
[1] 230
\end{verbatim}

\begin{Shaded}
\begin{Highlighting}[]
\CommentTok{\# Dados do problema instanciados como nós da árvore}
\NormalTok{dados}\OtherTok{=}\FunctionTok{as.Node}\NormalTok{(}\FunctionTok{yaml.load}\NormalTok{(stringr}\SpecialCharTok{::}\FunctionTok{str\_interp}\NormalTok{(}\StringTok{"}
\StringTok{name: Medicar}
\StringTok{type: decision}
\StringTok{VE: $\{VE.SIM\}}
\StringTok{Sim:}
\StringTok{    type: chance}
\StringTok{    desc: Doença}
\StringTok{    penwidth: 2}
\StringTok{    VE: $\{VE.SIM\}}
\StringTok{    A:}
\StringTok{        type: terminal}
\StringTok{        p: $\{pa\}}
\StringTok{        payoff: $\{v\_am\}}
\StringTok{        penwidth: 0.5}
\StringTok{    B:}
\StringTok{        type: terminal}
\StringTok{        p: $\{pb\}}
\StringTok{        payoff: $\{v\_bm\}}
\StringTok{        penwidth: 0.5}
\StringTok{Não:}
\StringTok{    type: chance}
\StringTok{    desc: Doença}
\StringTok{    payoff: {-}550}
\StringTok{    penwidth: 0.5}
\StringTok{    VE: $\{VE.NAO\}}
\StringTok{    A:}
\StringTok{        type: terminal}
\StringTok{        p: $\{pa\}}
\StringTok{        payoff: $\{v\_anm\}}
\StringTok{        penwidth: 0.5}
\StringTok{    B:}
\StringTok{        type: terminal}
\StringTok{        p: $\{pb\}}
\StringTok{        payoff: $\{v\_bnm\}}
\StringTok{        penwidth: 0.5}
\StringTok{"}\NormalTok{)))}
\end{Highlighting}
\end{Shaded}

Na sequência, definem-se as funções utilitárias para configurar os
estilos e demais detalhes da árvore de decisão.

\begin{Shaded}
\begin{Highlighting}[]
\CommentTok{\# Função para definir nome dos nós}
\NormalTok{GetNodeLabel }\OtherTok{=} \ControlFlowTok{function}\NormalTok{(dados) }\ControlFlowTok{switch}\NormalTok{(dados}\SpecialCharTok{$}\NormalTok{type, }\AttributeTok{terminal =} \FunctionTok{format}\NormalTok{(dados}\SpecialCharTok{$}\NormalTok{payoff, }\AttributeTok{scientific =} \ConstantTok{FALSE}\NormalTok{, }\AttributeTok{big.mark =} \StringTok{"."}\NormalTok{, }\AttributeTok{decimal.mark =} \StringTok{","}\NormalTok{),  }\FunctionTok{paste0}\NormalTok{(dados}\SpecialCharTok{$}\NormalTok{name, }\StringTok{"}\SpecialCharTok{\textbackslash{}n}\StringTok{["}\NormalTok{, dados}\SpecialCharTok{$}\NormalTok{VE, }\StringTok{"]"}\NormalTok{))}
\CommentTok{\# Função para definir valores nas arestas}
\NormalTok{GetEdgeLabel }\OtherTok{=} \ControlFlowTok{function}\NormalTok{(node) \{}
  \ControlFlowTok{if}\NormalTok{ (}\FunctionTok{isNotRoot}\NormalTok{(node) }\SpecialCharTok{\&\&}\NormalTok{ node}\SpecialCharTok{$}\NormalTok{parent}\SpecialCharTok{$}\NormalTok{type }\SpecialCharTok{==} \StringTok{\textquotesingle{}chance\textquotesingle{}}\NormalTok{) \{}
\NormalTok{    label }\OtherTok{=} \FunctionTok{paste0}\NormalTok{(node}\SpecialCharTok{$}\NormalTok{name, }\StringTok{" ("}\NormalTok{, node}\SpecialCharTok{$}\NormalTok{p, }\StringTok{")"}\NormalTok{)}
\NormalTok{  \} }\ControlFlowTok{else}\NormalTok{ \{}
\NormalTok{    label }\OtherTok{=}\NormalTok{ node}\SpecialCharTok{$}\NormalTok{name}
\NormalTok{  \}}
  \FunctionTok{return}\NormalTok{ (label)}
\NormalTok{\}}
\CommentTok{\# Funcão para definir o formato dos nós}
\NormalTok{GetNodeShape }\OtherTok{=} \ControlFlowTok{function}\NormalTok{(dados) }\ControlFlowTok{switch}\NormalTok{(dados}\SpecialCharTok{$}\NormalTok{type, }\AttributeTok{decision =} \StringTok{"box"}\NormalTok{, }\AttributeTok{chance =} \StringTok{"circle"}\NormalTok{, }\AttributeTok{terminal =} \StringTok{"none"}\NormalTok{)}
\CommentTok{\# Funcão para definir o estilo das arestas}
\NormalTok{GetArrowHead }\OtherTok{=} \ControlFlowTok{function}\NormalTok{(dados) }\ControlFlowTok{switch}\NormalTok{(dados}\SpecialCharTok{$}\NormalTok{type, }\AttributeTok{terminal =} \StringTok{"oinv"}\NormalTok{,  }\StringTok{"none"}\NormalTok{)}
\CommentTok{\# Funcão para definir espessura das arestas}
\NormalTok{GetPenWidth }\OtherTok{=} \ControlFlowTok{function}\NormalTok{(node) \{node}\SpecialCharTok{$}\NormalTok{penwidth\}}

\CommentTok{\# Definir os estilos}
\FunctionTok{SetEdgeStyle}\NormalTok{(dados, }\AttributeTok{fontname =} \StringTok{\textquotesingle{}helvetica\textquotesingle{}}\NormalTok{, }\AttributeTok{label=}\NormalTok{GetEdgeLabel, }\AttributeTok{arrowhead=}\NormalTok{GetArrowHead, }\AttributeTok{penwidth=}\NormalTok{GetPenWidth, }\AttributeTok{fontsize =} \DecValTok{9}\NormalTok{)}
\FunctionTok{SetNodeStyle}\NormalTok{(dados, }\AttributeTok{fontname =} \StringTok{\textquotesingle{}helvetica\textquotesingle{}}\NormalTok{, }\AttributeTok{label =}\NormalTok{ GetNodeLabel, }\AttributeTok{shape =}\NormalTok{ GetNodeShape, }\AttributeTok{fixedsize=}\NormalTok{T, }\AttributeTok{fontsize =} \DecValTok{9}\NormalTok{)}
\CommentTok{\# Renderizar a árvore}
\FunctionTok{ToDiagrammeRGraph}\NormalTok{(dados,}\AttributeTok{direction =} \StringTok{"climb"}\NormalTok{) }\SpecialCharTok{\%\textgreater{}\%}
  \FunctionTok{set\_global\_graph\_attrs}\NormalTok{(}\StringTok{"layout"}\NormalTok{, }\StringTok{"dot"}\NormalTok{, }\StringTok{"graph"}\NormalTok{) }\SpecialCharTok{\%\textgreater{}\%}
  \FunctionTok{add\_global\_graph\_attrs}\NormalTok{(}\StringTok{"rankdir"}\NormalTok{, }\StringTok{"LR"}\NormalTok{,}\StringTok{"graph"}\NormalTok{) }\SpecialCharTok{\%\textgreater{}\%}
  \FunctionTok{render\_graph}\NormalTok{()}
\end{Highlighting}
\end{Shaded}

Ao recomendar que o paciente tome a medicação, tem-se VE(Sim) = 230. Por
outro lado, ao recomendar que o paciente não tome a medicação, VE(Não) =
-550. Portanto, como VE(Sim) \textgreater{} VE(Não), VE = 230 e
\textbf{a recomendação a seguir é de que o paciente seja medicado}.

\begin{center}\rule{0.5\linewidth}{0.5pt}\end{center}

\hypertarget{b-calcular-o-vedip.}{%
\subsubsection{B) Calcular o VEdIP.}\label{b-calcular-o-vedip.}}

A incerteza envolvida neste problema é qual tipo de doença o paciente
tem (doença A ou B). Em uma situação em que se pode obter uma Informação
Perfeita, pode-se gerar a seguinte árvore, na qual a loteria não é mais
``Doença'' e sim ``Informante''.

\begin{Shaded}
\begin{Highlighting}[]
\CommentTok{\# Dados do problema instanciados como nós da árvore}
\NormalTok{dados}\OtherTok{=}\FunctionTok{as.Node}\NormalTok{(}\FunctionTok{yaml.load}\NormalTok{(stringr}\SpecialCharTok{::}\FunctionTok{str\_interp}\NormalTok{(}\StringTok{\textquotesingle{}}
\StringTok{name: Informante}
\StringTok{type: chance}
\StringTok{VE: $\{pa * v\_am + pb * v\_bnm\}}
\StringTok{penwidth: 0.5}
\StringTok{A:}
\StringTok{    name: Medicar}
\StringTok{    type: decision}
\StringTok{    penwidth: 0.5}
\StringTok{    desc: A}
\StringTok{    Medicar:}
\StringTok{        type: terminal}
\StringTok{        p: $\{pa\}}
\StringTok{        payoff: $\{v\_am\}}
\StringTok{        penwidth: 0.5}
\StringTok{B:}
\StringTok{    name: Não medicar}
\StringTok{    desc: B}
\StringTok{    type: decision}
\StringTok{    payoff: {-}550}
\StringTok{    penwidth: 0.5}
\StringTok{    VE: {-}550}
\StringTok{    Não medicar:}
\StringTok{        type: terminal}
\StringTok{        p: $\{pb\}}
\StringTok{        payoff: $\{v\_bnm\}}
\StringTok{        penwidth: 0.5}
\StringTok{\textquotesingle{}}\NormalTok{)))}
\CommentTok{\# Função para definir nome dos nós}
\NormalTok{GetNodeLabel }\OtherTok{=} \ControlFlowTok{function}\NormalTok{(dados) }\ControlFlowTok{switch}\NormalTok{(dados}\SpecialCharTok{$}\NormalTok{type, }\AttributeTok{terminal =} \FunctionTok{format}\NormalTok{(dados}\SpecialCharTok{$}\NormalTok{payoff, }\AttributeTok{scientific =} \ConstantTok{FALSE}\NormalTok{, }\AttributeTok{big.mark =} \StringTok{"."}\NormalTok{, }\AttributeTok{decimal.mark =} \StringTok{","}\NormalTok{),  dados}\SpecialCharTok{$}\NormalTok{name)}
\CommentTok{\# Função para definir valores nas arestas}
\NormalTok{GetEdgeLabel }\OtherTok{=} \ControlFlowTok{function}\NormalTok{(node) \{}
  \ControlFlowTok{if}\NormalTok{ (}\FunctionTok{isNotRoot}\NormalTok{(node) }\SpecialCharTok{\&\&}\NormalTok{ node}\SpecialCharTok{$}\NormalTok{parent}\SpecialCharTok{$}\NormalTok{type }\SpecialCharTok{==} \StringTok{\textquotesingle{}chance\textquotesingle{}}\NormalTok{) \{}
    \FunctionTok{return}\NormalTok{(}\FunctionTok{paste0}\NormalTok{(}\StringTok{"\&ldquo;"}\NormalTok{,node}\SpecialCharTok{$}\NormalTok{desc ,}\StringTok{"\&rdquo;"}\NormalTok{))}
\NormalTok{  \} }\ControlFlowTok{else} \ControlFlowTok{if}\NormalTok{ (}\FunctionTok{isNotRoot}\NormalTok{(node) }\SpecialCharTok{\&\&}\NormalTok{ node}\SpecialCharTok{$}\NormalTok{parent}\SpecialCharTok{$}\NormalTok{type }\SpecialCharTok{==} \StringTok{\textquotesingle{}decision\textquotesingle{}}\NormalTok{) \{}
    \FunctionTok{return}\NormalTok{()}
\NormalTok{  \}}
\NormalTok{\}}
\CommentTok{\# Funcão para definir o formato dos nós}
\NormalTok{GetNodeShape }\OtherTok{=} \ControlFlowTok{function}\NormalTok{(dados) }\ControlFlowTok{switch}\NormalTok{(dados}\SpecialCharTok{$}\NormalTok{type, }\AttributeTok{decision =} \StringTok{"box"}\NormalTok{, }\AttributeTok{chance =} \StringTok{"circle"}\NormalTok{, }\AttributeTok{terminal =} \StringTok{"none"}\NormalTok{)}
\CommentTok{\# Funcão para definir o estilo das arestas}
\NormalTok{GetArrowHead }\OtherTok{=} \ControlFlowTok{function}\NormalTok{(dados) }\ControlFlowTok{switch}\NormalTok{(dados}\SpecialCharTok{$}\NormalTok{type, }\AttributeTok{terminal =} \StringTok{"oinv"}\NormalTok{, }\AttributeTok{decision =} \StringTok{"normal"}\NormalTok{,  }\StringTok{"none"}\NormalTok{)}
\CommentTok{\# Funcão para definir espessura das arestas}
\NormalTok{GetPenWidth }\OtherTok{=} \ControlFlowTok{function}\NormalTok{(node) \{node}\SpecialCharTok{$}\NormalTok{penwidth\}}

\CommentTok{\# Definir os estilos}
\FunctionTok{SetEdgeStyle}\NormalTok{(dados, }\AttributeTok{fontname =} \StringTok{\textquotesingle{}helvetica\textquotesingle{}}\NormalTok{, }\AttributeTok{label=}\NormalTok{GetEdgeLabel, }\AttributeTok{arrowhead=}\NormalTok{GetArrowHead, }\AttributeTok{penwidth=}\NormalTok{GetPenWidth, }\AttributeTok{fontsize =} \DecValTok{7}\NormalTok{)}
\FunctionTok{SetNodeStyle}\NormalTok{(dados, }\AttributeTok{fontname =} \StringTok{\textquotesingle{}helvetica\textquotesingle{}}\NormalTok{, }\AttributeTok{label =}\NormalTok{ GetNodeLabel, }\AttributeTok{shape =}\NormalTok{ GetNodeShape, }\AttributeTok{fixedsize =}\NormalTok{ T, }\AttributeTok{fontsize =} \DecValTok{6}\NormalTok{, }\AttributeTok{width =} \FloatTok{0.5}\NormalTok{, }\AttributeTok{height =} \FloatTok{0.3}\NormalTok{)}
\CommentTok{\# Renderizar a árvore}
\FunctionTok{ToDiagrammeRGraph}\NormalTok{(dados,}\AttributeTok{direction =} \StringTok{"climb"}\NormalTok{) }\SpecialCharTok{\%\textgreater{}\%}
  \FunctionTok{set\_global\_graph\_attrs}\NormalTok{(}\StringTok{"layout"}\NormalTok{, }\StringTok{"dot"}\NormalTok{, }\StringTok{"graph"}\NormalTok{) }\SpecialCharTok{\%\textgreater{}\%}
  \FunctionTok{add\_global\_graph\_attrs}\NormalTok{(}\StringTok{"rankdir"}\NormalTok{, }\StringTok{"LR"}\NormalTok{,}\StringTok{"graph"}\NormalTok{) }\SpecialCharTok{\%\textgreater{}\%}
  \FunctionTok{render\_graph}\NormalTok{()}
\end{Highlighting}
\end{Shaded}

O Valor Esperado com Informação Perfeita (VEcIP) e Valor Esperado da
Informação Perfeita (VEdIP) podem ser calculados da seguinte maneira:

\begin{Shaded}
\begin{Highlighting}[]
\NormalTok{VEcIP }\OtherTok{\textless{}{-}}\NormalTok{ pa }\SpecialCharTok{*}\NormalTok{ v\_am }\SpecialCharTok{+}\NormalTok{ pb }\SpecialCharTok{*}\NormalTok{ v\_bnm}
\NormalTok{VEdIP }\OtherTok{=}\NormalTok{ VEcIP }\SpecialCharTok{{-}}\NormalTok{ VE}
\NormalTok{VEdIP}
\end{Highlighting}
\end{Shaded}

\begin{verbatim}
[1] 720
\end{verbatim}

Dessa forma, tem-se que \textbf{VEdIP = 720}.

\begin{center}\rule{0.5\linewidth}{0.5pt}\end{center}

\hypertarget{c-uxe9-possuxedvel-fazer-testes-e-esperar-para-eventual-inuxedcio-da-medicauxe7uxe3o.-o-custo-associado-a-estes-testes-incluindo-o-da-espera-em-si-uxe9-de-500.-a-eficiuxeancia-do-teste-uxe9-paa-85-e-pnana-95.-vale-a-pena-submeter-o-paciente-aos-testes-antes-de-decidir-pela-medicauxe7uxe3o-ou-nuxe3o-calcule-vedii.}{%
\subsubsection{C) É possível fazer testes e esperar para eventual início
da medicação. O custo associado a estes testes (incluindo o da espera em
si) é de 500. A eficiência do teste é P(``A''/A) = 85\% e P(``NA''/NA)=
95\%. Vale a pena submeter o paciente aos testes antes de decidir pela
medicação ou não? Calcule
VEdII.}\label{c-uxe9-possuxedvel-fazer-testes-e-esperar-para-eventual-inuxedcio-da-medicauxe7uxe3o.-o-custo-associado-a-estes-testes-incluindo-o-da-espera-em-si-uxe9-de-500.-a-eficiuxeancia-do-teste-uxe9-paa-85-e-pnana-95.-vale-a-pena-submeter-o-paciente-aos-testes-antes-de-decidir-pela-medicauxe7uxe3o-ou-nuxe3o-calcule-vedii.}}

Inicialmente, serão consideradas as informações de eficiência do teste
para calcular as probabilidades condicionais relacionadas ao resultado
do teste e as probabilidades que resultam das probabilidades \emph{a
priori} em conjunção com a qualidade da informação. Como o paciente
apresenta um conjunto de sintomas que pode caracterizar doença A ou B,
NA significa B. Por exemplo, P(``NA''\textbar NA) será representado por
P(``B''\textbar B).

\begin{Shaded}
\begin{Highlighting}[]
\NormalTok{pta\_a }\OtherTok{\textless{}{-}} \FloatTok{0.85}         \CommentTok{\# P("A"|A)}
\NormalTok{ptb\_b }\OtherTok{\textless{}{-}} \FloatTok{0.95}         \CommentTok{\# P("B"|B)}
\NormalTok{ptb\_a }\OtherTok{\textless{}{-}} \DecValTok{1} \SpecialCharTok{{-}}\NormalTok{ pta\_a    }\CommentTok{\# P("B"|A)}
\NormalTok{pta\_b }\OtherTok{\textless{}{-}} \DecValTok{1} \SpecialCharTok{{-}}\NormalTok{ ptb\_b    }\CommentTok{\# P("A"|B)}
\CommentTok{\# Dados \textquotesingle{}a priori\textquotesingle{} + Qualidade}
\NormalTok{pta }\OtherTok{\textless{}{-}}\NormalTok{ pa }\SpecialCharTok{*}\NormalTok{ pta\_a }\SpecialCharTok{+}\NormalTok{ pb }\SpecialCharTok{*}\NormalTok{ pta\_b  }\CommentTok{\# P("A")}
\NormalTok{ptb }\OtherTok{\textless{}{-}}\NormalTok{ pa }\SpecialCharTok{*}\NormalTok{ ptb\_a }\SpecialCharTok{+}\NormalTok{ pb }\SpecialCharTok{*}\NormalTok{ ptb\_b  }\CommentTok{\# P("B")}
\FunctionTok{cat}\NormalTok{(pta\_a, ptb\_b, ptb\_a, pta\_b, pta, ptb)}
\end{Highlighting}
\end{Shaded}

\begin{verbatim}
0.85 0.95 0.15 0.05 0.13 0.87
\end{verbatim}

Usando a regra de Bayes, podemos calculas as probabilidades \emph{a
posteriori.}

\begin{Shaded}
\begin{Highlighting}[]
\NormalTok{pa\_ta }\OtherTok{\textless{}{-}} \FunctionTok{round}\NormalTok{((pa }\SpecialCharTok{*}\NormalTok{ pta\_a) }\SpecialCharTok{/}\NormalTok{ pta, }\AttributeTok{digits =} \DecValTok{4}\NormalTok{)   }\CommentTok{\# P(A|"A")}
\NormalTok{pb\_ta }\OtherTok{\textless{}{-}} \FunctionTok{round}\NormalTok{((pb }\SpecialCharTok{*}\NormalTok{ pta\_b) }\SpecialCharTok{/}\NormalTok{ pta, }\AttributeTok{digits =} \DecValTok{4}\NormalTok{)   }\CommentTok{\# P(B|"A")}
\NormalTok{pa\_tb }\OtherTok{\textless{}{-}} \FunctionTok{round}\NormalTok{((pa }\SpecialCharTok{*}\NormalTok{ ptb\_a) }\SpecialCharTok{/}\NormalTok{ ptb, }\AttributeTok{digits =} \DecValTok{4}\NormalTok{)   }\CommentTok{\# P(A|"B")}
\NormalTok{pb\_tb }\OtherTok{\textless{}{-}} \FunctionTok{round}\NormalTok{((pb }\SpecialCharTok{*}\NormalTok{ ptb\_b) }\SpecialCharTok{/}\NormalTok{ ptb, }\AttributeTok{digits =} \DecValTok{4}\NormalTok{)   }\CommentTok{\# P(B|"B")}
\FunctionTok{cat}\NormalTok{(pa\_ta, pb\_ta, pa\_tb, pb\_tb)}
\end{Highlighting}
\end{Shaded}

\begin{verbatim}
0.6538 0.3462 0.0172 0.9828
\end{verbatim}

Após esses cálculos, é possível encontrar as demais informações
necessárias para construir a árvore de decisão. A seguir, também são
apresentados os cálculos do Valor Esperado com Informação Imperfeita
(\textbf{VEcII = 689}) e Valor Esperado da Informação Imperfeita
(\textbf{VEdII = 459}).

\begin{Shaded}
\begin{Highlighting}[]
\CommentTok{\# Valor esperado após o teste indicar "A" }
\NormalTok{v\_ta\_m  }\OtherTok{\textless{}{-}}  \FunctionTok{round}\NormalTok{(pa\_ta }\SpecialCharTok{*}\NormalTok{ v\_am }\SpecialCharTok{+}\NormalTok{ pb\_ta }\SpecialCharTok{*}\NormalTok{ v\_bm)}
\NormalTok{v\_ta\_nm }\OtherTok{\textless{}{-}}  \FunctionTok{round}\NormalTok{(pa\_ta }\SpecialCharTok{*}\NormalTok{ v\_anm }\SpecialCharTok{+}\NormalTok{ pb\_ta }\SpecialCharTok{*}\NormalTok{ v\_bnm)}
\NormalTok{VA      }\OtherTok{\textless{}{-}}  \FunctionTok{round}\NormalTok{(}\FunctionTok{max}\NormalTok{(v\_ta\_m, v\_ta\_nm))}

\CommentTok{\# Valor esperado após o teste indicar "B" }
\NormalTok{v\_tb\_m }\OtherTok{\textless{}{-}}   \FunctionTok{round}\NormalTok{(pa\_tb }\SpecialCharTok{*}\NormalTok{ v\_am }\SpecialCharTok{+}\NormalTok{ pb\_tb }\SpecialCharTok{*}\NormalTok{ v\_bm)}
\NormalTok{v\_tb\_nm }\OtherTok{\textless{}{-}}  \FunctionTok{round}\NormalTok{(pa\_tb }\SpecialCharTok{*}\NormalTok{ v\_anm }\SpecialCharTok{+}\NormalTok{ pb\_tb }\SpecialCharTok{*}\NormalTok{ v\_bnm)}
\NormalTok{VB }\OtherTok{\textless{}{-}} \FunctionTok{round}\NormalTok{(}\FunctionTok{max}\NormalTok{(v\_tb\_m, v\_tb\_nm))}

\CommentTok{\# Valor esperado com informação imperfeita}
\NormalTok{VEcII }\OtherTok{\textless{}{-}} \FunctionTok{round}\NormalTok{(pta }\SpecialCharTok{*}\NormalTok{ VA }\SpecialCharTok{+}\NormalTok{ ptb }\SpecialCharTok{*}\NormalTok{ VB)}
\NormalTok{VEcII}
\end{Highlighting}
\end{Shaded}

\begin{verbatim}
[1] 689
\end{verbatim}

\begin{Shaded}
\begin{Highlighting}[]
\CommentTok{\# Valor esperado da informação imperfeita}
\NormalTok{VEdII }\OtherTok{\textless{}{-}}\NormalTok{ VEcII }\SpecialCharTok{{-}}\NormalTok{ VE}
\NormalTok{VEdII}
\end{Highlighting}
\end{Shaded}

\begin{verbatim}
[1] 459
\end{verbatim}

Com essas informações, é possível gerar a árvore de decisão da seguinte
forma.

\begin{Shaded}
\begin{Highlighting}[]
\CommentTok{\# Dados do problema instanciados como nós da árvore}
\NormalTok{dados}\OtherTok{=}\FunctionTok{as.Node}\NormalTok{(}\FunctionTok{yaml.load}\NormalTok{(stringr}\SpecialCharTok{::}\FunctionTok{str\_interp}\NormalTok{(}\StringTok{"}
\StringTok{name: Testar}
\StringTok{desc: Testar}
\StringTok{type: decision}
\StringTok{VE: $\{VEcII\}}
\StringTok{Sim:}
\StringTok{    type: chance}
\StringTok{    desc: Teste}
\StringTok{    penwidth: 2}
\StringTok{    VE: $\{VEcII\}}
\StringTok{    }\SpecialCharTok{\textbackslash{}\textbackslash{}}\StringTok{\textquotesingle{}A}\SpecialCharTok{\textbackslash{}\textbackslash{}}\StringTok{\textquotesingle{}:}
\StringTok{        type: decision}
\StringTok{        desc: Medicar}
\StringTok{        VE: $\{VA\}}
\StringTok{        p: $\{pta\}}
\StringTok{        Sim:}
\StringTok{            type: chance}
\StringTok{            desc: Resultado}
\StringTok{            penwidth: 2}
\StringTok{            VE: $\{v\_ta\_m\}}
\StringTok{            A|}\SpecialCharTok{\textbackslash{}\textbackslash{}}\StringTok{\textquotesingle{}A}\SpecialCharTok{\textbackslash{}\textbackslash{}}\StringTok{\textquotesingle{}:}
\StringTok{                type: terminal}
\StringTok{                p: $\{pa\_ta\}}
\StringTok{                payoff: $\{v\_am\}}
\StringTok{                penwidth: 0.5}
\StringTok{            B|}\SpecialCharTok{\textbackslash{}\textbackslash{}}\StringTok{\textquotesingle{}A}\SpecialCharTok{\textbackslash{}\textbackslash{}}\StringTok{\textquotesingle{}:}
\StringTok{                type: terminal}
\StringTok{                p: $\{pb\_ta\}}
\StringTok{                payoff: $\{v\_bm\}}
\StringTok{                penwidth: 0.5}
\StringTok{        Não:}
\StringTok{            type: chance}
\StringTok{            desc: Resultado}
\StringTok{            payoff: {-}550}
\StringTok{            penwidth: 0.5}
\StringTok{            VE: $\{v\_ta\_nm\}}
\StringTok{            A|}\SpecialCharTok{\textbackslash{}\textbackslash{}}\StringTok{\textquotesingle{}A}\SpecialCharTok{\textbackslash{}\textbackslash{}}\StringTok{\textquotesingle{}:}
\StringTok{                type: terminal}
\StringTok{                p: $\{pa\_ta\}}
\StringTok{                payoff: $\{v\_anm\}}
\StringTok{                penwidth: 0.5}
\StringTok{            B|}\SpecialCharTok{\textbackslash{}\textbackslash{}}\StringTok{\textquotesingle{}A}\SpecialCharTok{\textbackslash{}\textbackslash{}}\StringTok{\textquotesingle{}:}
\StringTok{                type: terminal}
\StringTok{                p: $\{pb\_ta\}}
\StringTok{                payoff: $\{v\_bnm\}}
\StringTok{                penwidth: 0.5}
\StringTok{    }\SpecialCharTok{\textbackslash{}\textbackslash{}}\StringTok{\textquotesingle{}B}\SpecialCharTok{\textbackslash{}\textbackslash{}}\StringTok{\textquotesingle{}:}
\StringTok{        type: decision}
\StringTok{        desc: Medicar}
\StringTok{        VE: $\{VB\}}
\StringTok{        p: $\{ptb\}}
\StringTok{        Sim:}
\StringTok{            type: chance}
\StringTok{            desc: Resultado}
\StringTok{            penwidth: 0.5}
\StringTok{            VE: $\{v\_tb\_m\}}
\StringTok{            A|}\SpecialCharTok{\textbackslash{}\textbackslash{}}\StringTok{\textquotesingle{}B}\SpecialCharTok{\textbackslash{}\textbackslash{}}\StringTok{\textquotesingle{}:}
\StringTok{                type: terminal}
\StringTok{                p: $\{pa\_tb\}}
\StringTok{                payoff: $\{v\_am\}}
\StringTok{                penwidth: 0.5}
\StringTok{            B|}\SpecialCharTok{\textbackslash{}\textbackslash{}}\StringTok{\textquotesingle{}B}\SpecialCharTok{\textbackslash{}\textbackslash{}}\StringTok{\textquotesingle{}:}
\StringTok{                type: terminal}
\StringTok{                p: $\{pb\_tb\}}
\StringTok{                payoff: $\{v\_bm\}}
\StringTok{                penwidth: 0.5}
\StringTok{        Não:}
\StringTok{            type: chance}
\StringTok{            desc: Resultado}
\StringTok{            payoff: {-}550}
\StringTok{            penwidth: 2}
\StringTok{            VE: $\{v\_tb\_nm\}}
\StringTok{            A|}\SpecialCharTok{\textbackslash{}\textbackslash{}}\StringTok{\textquotesingle{}B}\SpecialCharTok{\textbackslash{}\textbackslash{}}\StringTok{\textquotesingle{}:}
\StringTok{                type: terminal}
\StringTok{                p: $\{pa\_tb\}}
\StringTok{                payoff: $\{v\_anm\}}
\StringTok{                penwidth: 0.5}
\StringTok{            B|}\SpecialCharTok{\textbackslash{}\textbackslash{}}\StringTok{\textquotesingle{}B}\SpecialCharTok{\textbackslash{}\textbackslash{}}\StringTok{\textquotesingle{}:}
\StringTok{                type: terminal}
\StringTok{                p: $\{pb\_tb\}}
\StringTok{                payoff: $\{v\_bnm\}}
\StringTok{                penwidth: 0.5}
\StringTok{Não:}
\StringTok{  type: decision}
\StringTok{  desc: Medicar}
\StringTok{  VE: $\{VE\}}
\StringTok{  Sim:}
\StringTok{      type: chance}
\StringTok{      desc: Doença}
\StringTok{      penwidth: 2}
\StringTok{      VE: 230}
\StringTok{      A:}
\StringTok{          type: terminal}
\StringTok{          p: $\{pa\}}
\StringTok{          payoff: $\{v\_am\}}
\StringTok{          penwidth: 0.5}
\StringTok{      B:}
\StringTok{          type: terminal}
\StringTok{          p: $\{pb\}}
\StringTok{          payoff: $\{v\_bm\}}
\StringTok{          penwidth: 0.5}
\StringTok{  Não:}
\StringTok{      type: chance}
\StringTok{      desc: Doença}
\StringTok{      payoff: {-}550}
\StringTok{      penwidth: 0.5}
\StringTok{      VE: {-}550}
\StringTok{      A:}
\StringTok{          type: terminal}
\StringTok{          p: $\{pa\}}
\StringTok{          payoff: $\{v\_anm\}}
\StringTok{          penwidth: 0.5}
\StringTok{      B:}
\StringTok{          type: terminal}
\StringTok{          p: $\{pb\}}
\StringTok{          payoff: $\{v\_bnm\}}
\StringTok{          penwidth: 0.5}
\StringTok{"}\NormalTok{)))}
\CommentTok{\# Função para definir nome dos nós}
\NormalTok{GetNodeLabel }\OtherTok{=} \ControlFlowTok{function}\NormalTok{(dados) }\ControlFlowTok{switch}\NormalTok{(dados}\SpecialCharTok{$}\NormalTok{type, }\AttributeTok{terminal =} \FunctionTok{format}\NormalTok{(dados}\SpecialCharTok{$}\NormalTok{payoff, }\AttributeTok{scientific =} \ConstantTok{FALSE}\NormalTok{, }\AttributeTok{big.mark =} \StringTok{"."}\NormalTok{, }\AttributeTok{decimal.mark =} \StringTok{","}\NormalTok{),  }\FunctionTok{paste0}\NormalTok{(dados}\SpecialCharTok{$}\NormalTok{desc, }\StringTok{"}\SpecialCharTok{\textbackslash{}n}\StringTok{["}\NormalTok{, dados}\SpecialCharTok{$}\NormalTok{VE, }\StringTok{"]"}\NormalTok{))}
\CommentTok{\# Função para definir valores nas arestas}
\NormalTok{GetEdgeLabel }\OtherTok{=} \ControlFlowTok{function}\NormalTok{(node) \{}
  \ControlFlowTok{if}\NormalTok{ (}\FunctionTok{isNotRoot}\NormalTok{(node) }\SpecialCharTok{\&\&}\NormalTok{ node}\SpecialCharTok{$}\NormalTok{parent}\SpecialCharTok{$}\NormalTok{type }\SpecialCharTok{==} \StringTok{\textquotesingle{}chance\textquotesingle{}}\NormalTok{) \{}
    \CommentTok{\# if (node$teste == True) \{}
    \CommentTok{\#   label = paste0(" (", node$name, ")")}
    \CommentTok{\# \}}
\NormalTok{    label }\OtherTok{=} \FunctionTok{paste0}\NormalTok{(node}\SpecialCharTok{$}\NormalTok{name, }\StringTok{" ("}\NormalTok{, node}\SpecialCharTok{$}\NormalTok{p, }\StringTok{")"}\NormalTok{)}
\NormalTok{  \} }\ControlFlowTok{else}\NormalTok{ \{}
\NormalTok{    label }\OtherTok{=}\NormalTok{ node}\SpecialCharTok{$}\NormalTok{name}
\NormalTok{  \}}
  \FunctionTok{return}\NormalTok{ (label)}
\NormalTok{\}}
\CommentTok{\# Funcão para definir o formato dos nós}
\NormalTok{GetNodeShape }\OtherTok{=} \ControlFlowTok{function}\NormalTok{(dados) }\ControlFlowTok{switch}\NormalTok{(dados}\SpecialCharTok{$}\NormalTok{type, }\AttributeTok{decision =} \StringTok{"box"}\NormalTok{, }\AttributeTok{chance =} \StringTok{"circle"}\NormalTok{, }\AttributeTok{terminal =} \StringTok{"none"}\NormalTok{)}
\CommentTok{\# Funcão para definir o estilo das arestas}
\NormalTok{GetArrowHead }\OtherTok{=} \ControlFlowTok{function}\NormalTok{(dados) }\ControlFlowTok{switch}\NormalTok{(dados}\SpecialCharTok{$}\NormalTok{type, }\AttributeTok{terminal =} \StringTok{"oinv"}\NormalTok{,  }\StringTok{"none"}\NormalTok{)}
\CommentTok{\# Funcão para definir espessura das arestas}
\NormalTok{GetPenWidth }\OtherTok{=} \ControlFlowTok{function}\NormalTok{(node) \{node}\SpecialCharTok{$}\NormalTok{penwidth\}}

\CommentTok{\# Definir os estilos}
\FunctionTok{SetEdgeStyle}\NormalTok{(dados, }\AttributeTok{fontname =} \StringTok{\textquotesingle{}helvetica\textquotesingle{}}\NormalTok{, }\AttributeTok{label=}\NormalTok{GetEdgeLabel, }\AttributeTok{arrowhead=}\NormalTok{GetArrowHead, }\AttributeTok{penwidth=}\NormalTok{GetPenWidth, }\AttributeTok{fontsize =} \DecValTok{9}\NormalTok{)}
\FunctionTok{SetNodeStyle}\NormalTok{(dados, }\AttributeTok{fontname =} \StringTok{\textquotesingle{}helvetica\textquotesingle{}}\NormalTok{, }\AttributeTok{label =}\NormalTok{ GetNodeLabel, }\AttributeTok{shape =}\NormalTok{ GetNodeShape, }\AttributeTok{fixedsize=}\NormalTok{T, }\AttributeTok{fontsize =} \DecValTok{8}\NormalTok{)}
\CommentTok{\# Renderizar a árvore}
\NormalTok{arvore }\OtherTok{\textless{}{-}} \FunctionTok{ToDiagrammeRGraph}\NormalTok{(dados, }\AttributeTok{direction =} \StringTok{"climb"}\NormalTok{) }\SpecialCharTok{\%\textgreater{}\%}
  \FunctionTok{set\_global\_graph\_attrs}\NormalTok{(}\StringTok{"layout"}\NormalTok{, }\StringTok{"dot"}\NormalTok{, }\StringTok{"graph"}\NormalTok{) }\SpecialCharTok{\%\textgreater{}\%}
  \FunctionTok{add\_global\_graph\_attrs}\NormalTok{(}\StringTok{"rankdir"}\NormalTok{, }\StringTok{"LR"}\NormalTok{,}\StringTok{"graph"}\NormalTok{)}
\FunctionTok{render\_graph}\NormalTok{(arvore, }\AttributeTok{width =} \DecValTok{500}\NormalTok{, }\AttributeTok{height =} \DecValTok{500}\NormalTok{)}
\end{Highlighting}
\end{Shaded}

Conforme calculado anteriormente, VEdII = 459. Considerando que o custo
associado ao teste (incluindo a espera) é de 500, \textbf{não vale a
pena submeter o paciente ao teste} uma vez que VEdII \textless{} 500.



\end{document}
